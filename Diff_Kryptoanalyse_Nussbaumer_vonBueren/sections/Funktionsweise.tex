\clearpage
\section{Funktionsweise}\label{sec:Funktionsweise}
Folglich soll die Funktionsweise einer differenziellen Kryptoanalyse auf den Data Encryption Standard (DES) erläutert werden. 
Wie der Name es schon andeutet, wird bei diesem Verfahren die Differenz aus zwei Klartexten, in diesem Beispiel mit $P_{1}$ und $P_{2}$ bezeichnet, verwendet. Die Differenz wird üblicherweise mit $P'$ bezeichnet und folgt aus einer XOR-Verknüpfung der Klartexte, also $P' = P_{1} \oplus P_{2}$.
Funktionen wie Expansionen, Permutationen oder XOR-Verknüpfen haben keine Einfluss auf die Differenz der Texte. Die Differenz kann also fast durch die gesamte Feistelstruktur beobachtet werden. Die Abbildung \ref{fig:DES_Differenz} zeigt wie sich eine Differenz durch das Netzwerk verhält. 

\begin{figure}[h]
	\centering
	\includegraphics[width=1.0\linewidth]{graphics/DES_Differenz.jpg}
	\caption{In dieser Tabelle sind die möglichen Risiken bewertet, sowie Präventionsmassnahmen definiert}
	\label{fig:DES_Differenz}
\end{figure}

Die wichtigste Eigenschaft bei der Spalte Differenz ist die doppelte XOR-Verknüpfung mit dem Schlüssel welche sich somit aufhebt. Mit anderen Worten ist: 

\begin{equation}\label{equ:Schluessel_Differenz}
S'_{in} = S_{in,1} \oplus S_{in,2} = (P_{E,1} \oplus K) \oplus (P_{E,2} \oplus K) = P_{E,1} \oplus P_{E,2} = P'_{E} 
\end{equation}


Wie bereits in der Einleitung erwähnt sind die S-Boxen nicht-lineraren Funktion. Um diese zu umgehen kann mit Wahrscheinlichkeiten gearbeitet werden.
Da die S-Boxen bekannt sind, kann eine Differenzenverteilungstabelle aufgestellt werden. In dieser wird für jede Eingangsdifferenz die Zahl Wertepaar gegeben welche eine bestimmte Ausgangsdifferenz erzeugen. Es gibt bei DES $2^{6} = 64$ verschieden mögliche Differenzen pro S-Box. Jede Eingangsdifferenz kann mit 64 verschiedenen Wertepaare erzeugt werden. Als Beispiel: für eine Eingangsdifferenz von $34_{h}$ (Hexadezimalzahl) gibt zwei Wertepaar die eine Ausgangsdifferenz von $04_{h}$ erzeugen beim Durchqueren der S-Boxen 1 \todo{Cite: Diff. Artikel!!}. 

Da bei einer chosen plaintext attack die Ausgangsdifferenz bekannt ist, können die möglichen Eingangswertepaare $(S_{in,1}, S_{in,2})$ in einer weiteren Tabelle abgelesen werden. Entsprechend der Differenzenverteilungstabelle gibt es mehr oder weniger solche möglichen Eingangspaar.
Für das Beispiel mit der Eingangsdifferenz $34_{h}$ und der Ausgangsdifferenz $04_{h}$ gibt es die 2 Wertepaare $(S_{in,1}, S_{in,2}) = (13_{h}, 27_{h})$ oder $(S_{in,1},S_{in,2}) = (27_{h}, 13_{h})$. Wäre die Ausgangsdifferenz $02_{h}$ bei einer Eingangsdifferenz von $34_{h}$ würde es 16 mögliche Wertepaare geben. 

Da nun die Eingangswerte bekannt sind, kann der Schlüssel wie folgt berechnet werden:
\begin{equation}\label{equ:Schluessel_Loesung}
K = P_{E,1} \oplus S_{in,1} = P_{E,2} \oplus S_{in,2}
\end{equation}

Da aber nicht mit Sicherheit gesagt werden kann welches Wertepaar $(S_{in,1}, S_{in,2})$ das richtige ist, gibt es bei diesem Beispiel zwei mögliche Schlüssel. Es müssten mehrere Durchgänge durchgeführt werden, um den falschen Schlüssel auszuschliessen. 














