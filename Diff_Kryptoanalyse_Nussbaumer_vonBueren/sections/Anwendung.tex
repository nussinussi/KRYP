%\clearpage
\section{Anwendung}\label{sec:Anwendung}
In der Funktionsweise wurde bis jetzt lediglich eine Runde von DES durchgelaufen. Es braucht nur wenige Klartext-Geheimtext Paare damit der Schlüssel für eine Runde ermittelt werden kann. 
Biham und Shamir haben weiter gezeigt, dass die Differenzielle Kryptoanalyse auf 2 Runden und mehr erweitert werden kann, doch umso höher die Anzahl Runden desto schwieriger die Entschlüsselung. Es muss mit den Wahrscheinlichkeiten aus der Differenzenverteilungstabelle weiter gerechnet werden. 
Bei 16 Runden wird die Wahrscheinlichkeit den richtigen Schlüssel zu finden so klein, dass ein Brut-force Attacke genau so schneller wäre.
Die differenzielle Kryptoanalyse ist hingegen effizient bei anderen DES-ähnliche Kryptosysteme. Verschlüsselungsverfahren wie die achtrunden Variante von Lucifer (Verschlüsselungsverfahren entworfen durch IBM vor DES) oder FEAL-4 / FEAL-8 können zum Beispiel mit der vorgestellten Methode gebrochen werden. FEAL mit weniger als 32 Runden können teilweise gebrochen werden \cite{biham_differential_1990}.















