%\clearpage
\section{Schluss}\label{sec:Schluss}
Mit der Differenziellen Kryptoanalyse wurde von Hiham und Shamir eine Methode veröffentlicht mit der ein statischer Angriff auf ein Verschlüsselungsalgorithmus durchgeführt werden kann. Wird der Angriff am Verschlüsselungsalgorithmus von Data Encryption Standart (DES) angewendet, wird in einem ersten Schritt der geheime Schlüssel durch zweifacher Anwendung der XOR-Verknüpfung aufgehoben. Nun können die nichtlinearen Funktionen (S-Boxen) analysiert werden. Mit den öffentlich zugänglichen S-Boxen werden Differenzentabellen erstellt in denen die Anzahl mögliche Wertepaare definiert sind, für die mit der Eingangsdifferenz eine Bestimmte Ausgangsdifferenz erzeugt werden kann. Aus den Eingangswertepaaren, welche die Differenz Ausgangsdifferenz erfüllen, können nun Mögliche Schlüssel mit XOR-Verknüpfungen, an Stelle wo der Schlüssel eingesetzt wird berechnet werden. Mit mehreren Runden von DES können nicht mehr alle wichtigen Differenzen ermittelt werden, da die Zwischenresultate nicht zur Verfügung stehen, es werden wiederholende Strukturen gesucht. Die Wahrscheinlichkeit den Richtigen Schlüssel zu finden nimmt ab. Bei 16 Runden wird die Wahrscheinlichkeit das richtige Paar zu finden so klein, dass eine Brut-force Attacke genauso schnell wäre. Bei anderen Verschlüsselungssysteme wie Lucifer oder FEAL kann die differenzielle Kryptoanalyse effizient angewendet werden. Seit dem die Methode der differenzielle Kryptoanalyse publiziert wurde, muss beim Entwurf eines Verschlüsselungsalgorithmus darauf geachtet werden, dass diese Methode nicht wirkungsvoll eingesetzt werden kann.









